% Derivation of R^2 and adjusted R^2
% _
% Author: Joram Soch, BCCN Berlin
% E-Mail: joram.soch@bccn-berlin.de
% Edited: 02/05/2019, 15:50


\setcounter{equation}{0}
\textbf{Index:} \\
$\vartriangleright$ The Book of Statistical Proofs \\
$\hphantom{\quad} \vartriangleright$ \Chaptername \\
$\hphantom{\quad} \hphantom{\quad} \vartriangleright$ \Sectionname


\paragraph{Theorem:}

Given a linear regression model

\begin{equation} \label{eq:rsq-mlr}
y = X\beta + \varepsilon, \; \varepsilon_i \overset{\mathrm{i.i.d.}}{\sim} \mathcal{N}(0, \sigma^2)
\end{equation}

with $n$ independent observations and $p$ independent variables,

1) the coefficient of determination is given by

\begin{equation} \label{eq:R2}
R^2 = 1 - \frac{\mathrm{RSS}}{\mathrm{TSS}}
\end{equation}

2) the adjusted coefficient of determination is

\begin{equation} \label{eq:R2-adj}
R^2_{\mathrm{adj}} = 1 - \frac{\mathrm{RSS}/(n-p)}{\mathrm{TSS}/(n-1)}
\end{equation}

where the residual and total sum of squares are

\vspace{-0.5em}
\begin{equation} \label{eq:SS}
\begin{split}
\mathrm{RSS} &= \sum_{i=1}^{n} (y_i - \hat{y}_i)^2, \quad \hat{y} = X\hat{\beta} \\
\mathrm{TSS} &= \sum_{i=1}^{n} (y_i - \bar{y})^2\;, \quad \bar{y} = \frac{1}{n} \sum_{i=1}^n y_i \\
\end{split}
\end{equation}

where $X$ is the $n \times p$ design matrix and $\hat{\beta}$ are the ordinary least squares estimates.


\paragraph{Proof:} The coefficient of determination $R^2$ is defined as the proportion of the variance explained by the independent variables, relative to the total variance in the data.

\vspace{1em}
1) If we define the explained sum of squares as

\begin{equation} \label{eq:ESS}
\mathrm{ESS} = \sum_{i=1}^{n} (\hat{y}_i - \bar{y})^2 \; ,
\end{equation}

then $R^2$ is given by

\begin{equation} \label{eq:R2-s1}
R^2 = \frac{\mathrm{ESS}}{\mathrm{TSS}} \; .
\end{equation}

which is equal to

\begin{equation} \label{eq:R2-s2}
R^2 = \frac{\mathrm{TSS}-\mathrm{RSS}}{\mathrm{TSS}} = 1 - \frac{\mathrm{RSS}}{\mathrm{TSS}} \; ,
\end{equation}

because $\mathrm{TSS} = \mathrm{ESS} + \mathrm{RSS}$. \hspace\fill $\square$

\vspace{1em}
2) Using (\ref{eq:SS}), the coefficient of determination can be also written as:

\begin{equation} \label{eq:R2'}
R^2 = 1 - \frac{\sum_{i=1}^{n} (y_i - \hat{y}_i)^2}{\sum_{i=1}^{n} (y_i - \bar{y})^2} = 1 - \frac{\frac{1}{n} \sum_{i=1}^{n} (y_i - \hat{y}_i)^2}{\frac{1}{n} \sum_{i=1}^{n} (y_i - \bar{y})^2} \; .
\end{equation}

If we replace the variance estimates by their unbiased estimators, we obtain

\begin{equation} \label{eq:R2-adj'}
R^2_{\mathrm{adj}} = 1 - \frac{\frac{1}{n-p} \sum_{i=1}^{n} (y_i - \hat{y}_i)^2}{\frac{1}{n-1} \sum_{i=1}^{n} (y_i - \bar{y})^2} = 1 - \frac{\mathrm{RSS}/\mathrm{df}_r}{\mathrm{TSS}/\mathrm{df}_t}
\end{equation}

where $\mathrm{df}_r = n-p$ and $\mathrm{df}_t = n-1$ are the residual and total degrees of freedom. \hspace\fill $\square$

This gives the adjusted $R^2$ which adjusts $R^2$ for the number of explanatory variables. \hspace\fill $\blacksquare$


\paragraph{Dependencies:}
\begin{itemize}
\item ordinary least squares for multiple linear regression
\item total, explained and residual sum of squares
\item unbiased estimator for the residual variance
\end{itemize}


\paragraph{Source:}
\begin{itemize}
\item Wikipedia: "Coefficient of determination"; in: \textit{Wikipedia, the free encyclopedia}; URL: \url{https://en.wikipedia.org/wiki/Coefficient_of_determination#Adjusted_R2}.
\end{itemize}


\renewcommand{\Meta}[4]{
\paragraph{Metadata:} ID: #1 $|$ name: #2 $|$ author: #3 $|$ date: #4.
}
\Meta{A008}{glm-mle}{JoramSoch}{2019-05-07}