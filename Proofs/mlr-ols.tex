% Ordinary least squares for multiple linear regression
% _
% Author: Joram Soch, BCCN Berlin
% E-Mail: joram.soch@bccn-berlin.de
% Edited: 02/05/2019, 15:50


\setcounter{equation}{0}
\textbf{Index:} \\
$\vartriangleright$ The Book of Statistical Proofs \\
$\hphantom{\quad} \vartriangleright$ \Chaptername \\
$\hphantom{\quad} \hphantom{\quad} \vartriangleright$ \Sectionname


\paragraph{Theorem:}

Given a linear regression model with independent observations

\begin{equation} \label{eq:mlr}
y = X\beta + \varepsilon, \; \varepsilon_i \overset{\mathrm{i.i.d.}}{\sim} \mathcal{N}(0, \sigma^2) \; ,
\end{equation}

the parameters minimizing the residual sum of squares are given by

\begin{equation} \label{eq:mlr-ols}
\hat{\beta} = (X^\mathrm{T} X)^{-1} X^\mathrm{T} y \; .
\end{equation}


\paragraph{Proof:} Let $\hat{\beta}$ be the ordinary least squares (OLS) solution and let $\hat{\varepsilon} = y - X\hat{\beta}$ be the resulting vector of residuals. Then, this vector must be orthogonal to the design matrix,

\begin{equation} \label{eq:X-e-orth}
X^\mathrm{T} \hat{\varepsilon} = 0 \; ,
\end{equation}

because if it wasn't, there would be another solution $\tilde{\beta}$ giving another vector $\tilde{\varepsilon}$ with a smaller residual sum of squares. From (\ref{eq:X-e-orth}), the OLS formula can be directly derived:

\vspace{-0.5em}
\begin{equation} \label{eq:mlr-ols-proof}
\begin{split}
X^\mathrm{T} \hat{\varepsilon} &= 0 \\
X^\mathrm{T} \left( y - X\hat{\beta} \right) &= 0 \\
X^\mathrm{T} y - X^\mathrm{T} X\hat{\beta} &= 0 \\
X^\mathrm{T} X\hat{\beta} &= X^\mathrm{T} y \\
\hat{\beta} &= (X^\mathrm{T} X)^{-1} X^\mathrm{T} y \; .
\end{split}
\end{equation}

\vspace{-1em}
\hspace\fill $\blacksquare$


\paragraph{Source:}
\begin{itemize}
\item Stephen, Klaas Enno (2010): "The General Linear Model (GLM)"; in: \textit{Methods and models for fMRI data analysis in neuroeconomics}; URL: \url{http://www.socialbehavior.uzh.ch/teaching/methodsspring10.html}
\end{itemize}


\paragraph{Index:} number: A002; shortcut: mlr-ols; author: JoramSoch; date: 2019/05/02.